\documentclass[a4paper]{article}

\usepackage{proof}
\usepackage{mathtools}
\usepackage{listings}
\usepackage{rotating}
\usepackage[dvipsnames]{xcolor}

\lstdefinelanguage{AppPAL}{%
  morekeywords={if,inf,says,where,true,false,with,is},
  otherkeywords={can-act-as,can-say},
  sensitive=true,
  morestring=[b]',
  literate={\ inf\ }{{$\infty$}}5
}[keywords,strings]
% Margin listings
\lstset{%
  basicstyle=\ttfamily,
  stringstyle=\sffamily,
  keywordstyle=\ttfamily\slshape,
  language=AppPAL,
  columns=flexible,
  mathescape,
}

\newcommand{\new}[1]{{\color{BrickRed} #1}}
\newcommand{\ac}[0]{\ensuremath\text{AC}}
\newcommand{\secpalmath}[1]{\ensuremath\texttt{#1}}
\newcommand{\says}[1]{\secpalmath{says}^{\new{#1}}}
\newcommand{\canSay}[1]{\secpalmath{can-say}_{#1}}
\newcommand{\canActAs}[0]{\secpalmath{can-act-as}}
\newcommand{\spif}[0]{\secpalmath{if}}
\newcommand{\where}[0]{\secpalmath{where}}

\title{Plausable SecPAL}
\author{Joseph Hallett}
\begin{document}
\maketitle

The SecPAL authorization language, and the AppPAL instantiation, allow policy authors to make use of static analysis tools to make decisions, and allow principals to make statements about apps through delegation.
When these decisions are made they are made with certainty.  If a principal says an app is safe to access the network then we believe that that principal definitely believes the app is safe on a network.
When a static analysis tool finds that an app isn't malware then we believe that app to not be malware.
This isn't realistic.  
Static analysis tools can produce false results.  
A principal might be merely fairly confident that an app can access the network but not absolutely certain.

With current authorization languages, such as AppPAL and XACML, there is no way to quantify the belief a principal has in any statement.
A principal cannot say how plausable they think any statement is.

\section{Why do we need this?}

SecPAL was designed for making access control decisions.
The decision whether to install allow a user access to a file or not is a binary one: either they can access it or they cannot.
Similarly the decision process for these decisions is also binary: a user is either logged in or not, a network address is either in the network or outside it, someone can act as someone else's manager or they can not.

AppPAL, however, is primarily for deciding what apps you want to use.
Whether you want to install an app or not is less binary than an access control decision, because it is ultimately an opinion.
Consider a really simple policy that says \emph{``do not install malware''}:
  you could try using a malware scanner to check apps, but there opinions on apps can change rapidly and often.
You could use a meta-scanning tool like \emph{VirusTotal}, but then you'd only get the percentage of antivirus tools that flagged the app as malicious.
Without plausability we cannot represent the doubt and confidence in any assertion.

For another example consider a policy you only want to install apps that are made by \emph{reputable} developers and that are \emph{safe}.
Both reputable and safe are poorly defined, and badly represented by a binary decision.
A developer may be reputable if they are a large developer with a lot of staff like Google or Facebook, they might be reputable if their apps have been well reviewed, but what about a developer like \emph{King} who produce a large number of games with in-app-purchases, TV adverts, and sketchy privacy records.  They are probably more reputable than a malware author but you might have less confidence that they are producing good apps than Google.  Similarly an app might be seen as safe if it doesn't request any permissions and has no native code, but if it starts requesting more permissions and the amount of native code grows then the plausability it is safe should fall. 
When combined into the policy as a whole Google may be able to get away with a lot more permissions than other developers simply because we trust them more not to be evil.
Again, without plausability it is difficult to represent these decisions.

\section{Bending the Rules}

SecPAL has three rules for evaluation: \emph{cond}, \emph{can-say}, and \emph{can-act-as}.
We modify the language so that the \emph{says} keyword has an an annotation $0 \geq p \geq 1$ denoting a statements \emph{plausability}.
If the annotation is missing then it is assumed to be $1$
We also assume a plausability combining function $\oplus$ which combines plausability.
The SecPAL inference rules then become as follows, with additions highlighted in \new{red}. 

{\footnotesize\centering
\begin{eqnarray}
  \infer[\text{cond\new{$\leq$}}]{
    \ac, D \models A~\says{\bigoplus_{i=1}^n p_i}~f\theta
  }{
    \begin{matrix}{
      \left(A~\says{\text{at least}~p_{lim}}~f~if~f_1\cdots f_n~\where~c\right) \in \ac
    }\\{
      \forall i \in [1\cdots n]. \ac, D \models A~\says{p_i}~f_i\theta
    }\\\new{
      0 < p_{lim} \leq \bigoplus_{i=1}^n p_i
    }
    \end{matrix}&
    \vdash c\theta &
    vars\left(f\theta\right) = \emptyset
  }
  \\
  \infer[\text{cond\new{=}}]{
    \ac, D \models A~\says{p_{lim}}~f\theta
  }{
    \begin{matrix}{
      \left(A~\says{\text{is}~p_{lim}}~f~if~f_1\cdots f_n~\where~c\right) \in \ac
    }\\{
      \forall i \in [1\cdots n]. \ac, D \models A~\says{p_i}~f_i\theta
    }\\\new{
      0 < p_{lim} \leq \bigoplus_{i=1}^n p_i
    }
    \end{matrix}&
    \vdash c\theta &
    vars\left(f\theta\right) = \emptyset
  }
  \\
  \infer[\text{can-say}]{
    \ac, \infty \models A~\says{p_1 \oplus p_2}~f
  }{
    \ac, \infty \models A~\says{p_1}~B~\canSay{D}~f &
    \ac, D \models B~\says{p_2}~f
  }
  \\
  \infer[\text{can-act-as}]{
    \ac, D \models A~\says{p_1 \oplus p_2}~x~vp
  }{
    \ac, D \models A~\says{p_1}~x~\canActAs~y &
    \ac, D \models B~\says{p_2}~y~vp
  }
  \\
  \new{
    \infer[\text{reduce}]{
        \ac, D \models A~\says{p}~f
    }{
        \ac, D \models A~\says{p^\prime}~f & p \leq p^\prime
    }
  }
\end{eqnarray}
}

In general any derived statement is at most as plausible as the combination of the statements that went into deriving it.  We split the \emph{cond} rule into two variants.  The \emph{cond$\leq$} rule allows us to specify a minimum plausability required by combining all the conditional statements and if that limit is exceeded we take the combined probability in the plausability of the outcome.
The \emph{cond=} rule allows us again to set a minimum plausability but this time we take the stated plausability if the rule is satisfied.  These two \emph{cond} rules serve different purposes.  The \emph{cond=} variant is useful when we want to set a limit on the plausability: for instance when we have a tool with a known confidence rate we want to run, or a fact which we know how plausible it is.  The \emph{cond$eq$} rule is useful for when you want to ensure that a decision is made with a certain least-confidence, for instance if you want to be at least 80\% sure that an app is safe to use before doing anything with it.  In this case we would want the combined plausability to trickle through the proof not the lower limit.

We also add a plausability reduction rule that allows us to reduce the plausability of an assertion, this allows us to phrase a policy query as \emph{``is it at least 50\% plausable that...''} rather than having to discover the plausabilities precisely.

\subsection{Plausability Combination}

How should the plausability combination operator be defined?
One simplistic approach might be to take the least plauable assertion as the final value.
This however is too simple. Consider the case where an app needs to be safe and from a reputable developer.  Say we're 50\% sure the app is safe but we are unsure of the developer. We are 90\% sure Google is reputable and only 60\% sure King is.  No matter who the developer is the combined probability is the same (50\%) so the distinction is lost.

\begin{equation}
  p_a~\oplus_{\text{min}}~p_b~\gets~\text{min}~p_a~p_b
\end{equation}

If we are sure the plausabilities are independent we could multiply them together to get the final plausability. This is appealing in that if we take the example above if Google is the developer we're 45\% sure the app is usable, but only 30\% if King is.  If however there are a lot of combinations to be made the probabilities may get very small and be difficult to distinguish.

\begin{equation}
  p_a~\oplus_{\text{ind}}~p_b~\gets~p_a\times p_b
\end{equation}

What we \emph{probably} want is some form of \emph{Dempster-Shafer} belief combination algorithm.  Seems to be a fair bit of work here from the Expert Systems perople but I can't see them being applied to authorizatin logics before.

\textbf{TODO Learn more about them}

\section{What Guarantees Do We Want?} 

\begin{itemize}
\item If all statements have a plausability of $1$, then this is equivalent to standard SecPAL.
\item If a statement has a plauability of $0$, then it is equivalent to the statement not existing in the assertion context.
\item No statement should be more certain than the conditionals used to derive it.
\end{itemize}

Point 1 is true since if $p_{\text{lim}} = 1$ then no statement will be accepted unless its plausability is also 1.
If we combine the plausability of two events this will also be 1 (using either method described so far):
    hence the restriction on the plausability in the cond rule is always true since all statements have a plausability of one.
Rewriting the rules with this in mind we get the original SecPAL inference rules, so if all statements have a plausability of 1, then this must be equivalent to standard SecPAL.

Point 2 is also true by the restriction on probabilities in the cond rule.
Since $p_{\text{lim}}$ must be greater than 0, if a statement has a plausability of 0 then it will not be accepted.
Similarly when combining plausabilities if a statement is completely inplausible then the plausibility of it happening with another event must also be implausible.
Hence the combination should always be 0, and no statement combined with an implausible statement will be derivable.

Point 3 is important as we don't want a means to make a statement more certain
by repeatedly applying a rule.  For example imagine we had a combination
operator that took the sum (or 1 if the combination was greater than 1), and a
rule such as:
\begin{lstlisting} 
'x' says 'y' p if 'y' p, 'y' p.
\end{lstlisting}
If we also have statement that $x \says{0.2} y p$, then we
could apply this rule to get $x \says{0.4} y p$, and so on until we're certain.

Similarly if we had statements  such as:
\begin{lstlisting} 
'x' says 'y' p 
  if 'y' p,
     'z' p.
\end{lstlisting}
And we knew that $'x'\says{1.0}{} 'z' p$, if we used a plausability combination function where we took the average then we could again increase the certainty of $'x'\says{}{} 'y' p.$.

The multiplication and minimum plausability rules won't allow you to increase plausability like this (minimum is trivial: if it is always the plausability of the least plausible statement then plausabilities will never increase; multiplication works because if all plausabilities are between 1 and 0, then the product will also never increase).

\section{Examples}

\subsection{Tools with Confidences}

Suppose we have a static analysis tool with a false positive rate of 10\%, (i.e.~one app in ten it will falsely flag as having issues when infact it is fine).

\begin{lstlisting}
'user' says App isSafe
  if App isAnApp
  where staticAnalyisTool(App) = true
  with plausability is 0.9.
\end{lstlisting}

This is useful because we might want to treat static analysis results, along with other information, as more reliable.  For instance we might say an app which isn't malware is good, but an app that isn't malware and is recommended by a friend is even better!

\begin{lstlisting}
'user' says App isGood
  if App isntMalware
  with plausability is 0.5.

'user' says app isGood
  if App isntMalware,
     App isRecommended
  with plausability is 0.8.
\end{lstlisting}

\subsection{Review scores}

When picking games we may rely on expert reviewers to suggest apps to us.
It is very common to give reviewed items a score at the end of the review.

\begin{lstlisting}
'less-trusted-reviewer' says 'angry-birds' isGood
  with plausability is 0.8.

'trusted-reviewer' says 'angry-birds' isGood
  with plausability is 0.9.
\end{lstlisting}

We might also trust some reviewers more than others, again we can express this screnario with plausability and attempt to normalize their scores somewhat.

\begin{lstlisting}
'user' says 'trusted-reviewer' can-say App isGood
  if App isAnApp
  with plausability is 1.0.

'user' says 'less-trusted-reviewer' can-say App isGood
  if App isAnApp
  with plausability is 0.5.
\end{lstlisting}

When we come to picking an app though we might like to rely on multiple bits of information however.

\begin{lstlisting}
'user' says App isInstallable
  if App isGood,
     App isSafe
  with plausability at least 0.75.
\end{lstlisting}

\section{Translation to Datalog (with constraints, and bounded numbers)}

To evaluate this, it would be nice to show, (as Becker did) that these rules are reducible to some form of Datalog.
This would show we haven't mucked about with the decidability of SecPAL.
To do this I show an updated form of Becker's Algorithm 5.2 from the SecPAL reference document:

\subsection{A Plausible Algorithm 5.2}

We now describe an algorithm for translating an assertion context into an equivalent
constrained Datalog program. We treat expressions of the form $e_1 says_k fact$ as Datalog
literals, where $k$ is either a variable or 0 or $\infty$. This can be seen as a sugared notation for
a literal where the predicate name is the string concatenation of all infix operators (\textsf{says},
\textsf{can-say}, \textsf{can-act-as}, and predicates) occurring in the expression, including subscripts for
\textsf{can-say}. The arguments of the literal are the collected expressions between these infix
operators. For example, the expression $A~says_k^{\new{p}}~x~can say_\infty~y~can say_0~B~can act as~z$ is
shorthand for \texttt{says\_cansay\_infinity\_cansay\_zero\_canactas(A, \new{p}, k, x, y, B, z)}.

Given an assertion: 
\begin{center} \lstinline!$A$ says $f_0$ if $f_1\cdots f_n$ where $c$ with plausability $p$.! \end{center}

\begin{enumerate}
\item 
  If $f_0$ is flat (it isn't a can-say statement), then the assertion is translated into the clause:
  \begin{lstlisting}[language=Prolog]
$A$ says$_k^{\new{p_*}}$ $f_0$ :- 
    $A$ says$_k^{\new{p_1}}$ $f_1$ $\cdots$ $A$ says$_k^{\new{p_n}}$ $f_n$, c, 
    $\new{p_\Sigma \text{ is } p_1 \oplus \cdots \oplus p_n}$, 
    $\new{0 < p_{lim} \leq p_\Sigma}$.
  \end{lstlisting}
  Where $k$ is a fresh variable \new{and $p_*$ is $p_{lim}$ if the plausability is \texttt{is}, and $p_\Sigma$ is it is \texttt{at least}}.
  
\item 
  Otherwise $f_0$ is of the form \lstinline!$e_0$ can-say $D_0$ $\cdots$ $e_{n-1}$ can-say $D_{n-1}$ $f$! where $f$ is flat.
  Let $f^\prime_n \equiv f$ and \lstinline!$f^\prime_i \equiv e_i$ can-say $D_i$ $f^\prime_{i+1}$!, for $i\in\left\{0\cdots n-1\right\}$.
  Note that $f_0 = f^\prime_0$.  

  Then the assertion \lstinline!$A$ says $f_0$ if $f_1\cdots f_m$, c, with plausability $p$! is translated into a set of $n+1$ Datalog rules as follows.
  
  \begin{enumerate}
  \item 
    We add the Datalog rule:
    \begin{lstlisting}[language=Prolog]
$A$ says$_k^{\new{p_*}}$ $f^\prime_0$ :-
    $x$ says$_k^{\new{p_1}}$ $f_1\cdots$ $A$ says$_k^{\new{p_m}}$ $f_m$, c,
    $\new{p_\Sigma \text{ is } p_1 \oplus \cdots \oplus p_n}$, 
    $\new{0 < p_{lim} \leq p_\Sigma}$.
    \end{lstlisting}
    Where $k$ is a fresh variable, \new{and $p_*$ is $p_{lim}$ if the plausability is \texttt{is}, and $p_\Sigma$ is it is \texttt{at least}}.

  \item
    For each $i\in\left\{1\cdots n\right\}$, we add a Datalog rule
    \begin{lstlisting}[language=Prolog]
$A$ says$_\infty^{\new{p_*}}$ $f^\prime_i$ :-
    $x$ says$_{D_{i-1}}^{\new{p_1}}$ $f^\prime_i$,
    $A$ says$_{\infty}^{\new{p_2}}$ $x$ can-say $D_{i-1}$ $f^\prime_i$,
    $\new{p_* \text{ is } p_1\oplus p_2}$, 
    $\new{0 < p_* \leq 1}$.
    \end{lstlisting}
    Where $x$ is a fresh variable.
  \end{enumerate}
  
  \item
    For each Datalog rule created above of the form:
    \begin{lstlisting}[language=Prolog]
      $A$ says$_k^p$ $e$ $v$ :- $\cdots$
    \end{lstlisting}
    we add a rule:

    \begin{lstlisting}[language=Prolog]
$A$ says$_\infty^{\new{p_*}}$ $e$ $v$ :-
    $x$ says$_{k}^{\new{p_1}}$ $x$ can-act-as $e$,
    $A$ says$_{k}^{\new{p_2}}$ $e$ $v$,
    $\new{p_* \text{ is } p_1\oplus p_2}$, 
    $\new{0 < p_* \leq 1}$.
    \end{lstlisting}
    Where $x$ is a fresh variable.  Note that $k$ is not a fresh variable, but either a constant or a variable taken from the original rule.
    

    \new{
      We also add an additional rule (to account for the reduce rule)
      that should not be used in general, but only when trying to reduce the
      plausability to account for a lower bound on plausability in a query:
      }
      
      \begin{lstlisting}[basicstyle=\color{BrickRed}\ttfamily]
$A$ says$_k^{p_\downarrow}$ $e$ $v$ :- 
    $A$ says$_k^p$ $e$ $v$,
    $p_\downarrow \leq p$.
      \end{lstlisting}
\end{enumerate}

\end{document}
-